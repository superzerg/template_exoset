%------------------
% Mise en page
%------------------

%\textheight 230mm             %     Einstellungen fuer A4 Papier
\textwidth 168mm              %
\topmargin -1cm 
\oddsidemargin -4mm \evensidemargin -4mm 
%\pagestyle{plain}
\parindent 0mm

\usepackage{fancyhdr}  % mise en page fancy
\usepackage{lastpage}   % permet de se référer au nombre de page total du document
\pagestyle{fancy}
\fancyhf{}

% clear any old style settings
\fancyhead{}
\fancyfoot{}

% Header
\setlength{\headheight}{60pt}
\renewcommand\headrulewidth{3pt}
\fancyhead[L]{École Polytechnique Fédérale de Lausanne}
\fancyhead[R]{\includegraphics[width=0.08\linewidth]{../cartouche/2000px-Logo_EPFL.png} }


% Footer
\renewcommand\footrulewidth{1pt}
\fancyfoot[L]{Recueil de problémes de physique générale}

\fancyfoot[R]{\thepage/\pageref{LastPage}}

\usepackage{epsfig}
\usepackage{amsmath}
\usepackage{calc}
% \usepackage[T1]{fontenc}
\usepackage{url}
\usepackage{cancel}
\usepackage{graphicx,color}
\usepackage{xcolor}
\usepackage{lipsum}

% \usepackage[latin1]{inputenc} % entree 8 bits iso-latin1
% \usepackage[T1]{fontenc}      % encodage 8 bits des fontes utilisées
%-----------------
% TIKZ et PGF
%-----------------
\usepackage{pgf}			% PGF is a macro packéage for creéatéing graphéics. It comes with a user-friendly synétax layer called TikZ
\usepackage{tikz}			
\usepackage{pgfplots}		% PGFPlots draws high-qualéity funcétion plots with a user-friendly inéteréface diérectly in TeX.
\pgfplotsset{compat=1.12}

\tikzset{math3d2/.style=
    {x= {(-0.3cm,-0.42cm)}, z={(0cm,1cm)},y={(.7cm,0cm)}}} % Axonométrie: x sur l'avant gauche, y en face
    