%--------------------------------------------------------------------------
%
%                                                    SOLUTION
%
%--------------------------------------------------------------------------

\begin{center}
\vspace*{5mm}
\noindent {\Large {\bf (Solution) }}
\end{center}

A diagram synthesising all available information is given in the problem. We add to that diagram a $Oxy$ frame, where the origin $O$ is the fox's position, and the angle $\alpha$, which shows the direction of $v_0$ with respect to the $x$-axis.
\begin{center}
\input{figures/serie01_c_fig3.pdf_t}
\end{center}
\begin{enumerate}
\item[a)] The initial conditions are\footnote{For ease of writing, $||\vec{v}_0||$ is simply written $v_0$}:\\
\begin{itemize}
\item for the stone: \\
\begin{displaymath}
\textrm{position:}
\left\{ \begin{array}{l}
x_0^S = 0 \\
y_0^S = 0
\end{array}\right.\qquad
\textrm{velocity:}
\left\{ \begin{array}{l}
v_{x,0}^S = v_0\cos\alpha\\
v_{y,0}^S = v_0\sin\alpha
\end{array}\right.\qquad
\end{displaymath}
\item for the cheese: \\
\begin{displaymath}
\textrm{position:}
\left\{ \begin{array}{l}
x_0^C = L \\
y_0^C = H
\end{array}\right.\qquad
\textrm{velocity:}
\left\{ \begin{array}{l}
v_{x,0}^C = 0\\
v_{y,0}^C = 0.
\end{array}\right.\qquad
\end{displaymath}
\end{itemize}
\item Since both considered systems (system 1: the stone and system 2: the cheese) are only subject to gravity, their equations of motions are given by Newton's second law : 
\[
\ddot{\vec{r}}^S(t) = \vec{g} \qquad \textrm{et} \qquad \ddot{\vec{r}}^C(t) = \vec{g} \qquad \textrm{avec} \qquad \vec{r}(t)=\left(\begin{array}{c}x(t)\\y(t)\\z(t)\end{array}\right)
\]
Projecting both of these equations on the $Oxy$ frame, we get four equations of motion:\\
\begin{itemize}
\item Equation for the stone, projected onto the $Ox$ horizontal axis: 
\[
\ddot{x}^S(t) = 0.
\]
Let's integrate this once:
\[
\dot{x}^S(t) = cte.
\]
The initial condition  $v_{x,0}^S  \equiv \dot{x}^S(t=t_0) = v_0\cos\alpha$ allows to compute the constant
\[
\dot{x}^S(t) = v_0\cos\alpha.
\]
To find the position, we integrate a second time: 
\[
x^S(t)=(v_0\cos\alpha) t + cte'.
\]
The initial condition $x_0^S = 0$ gives us $cte' = 0$, and so
\[
x^S(t)=(v_0\cos\alpha) t.
\]\\
\item Equation for the stone, projected on $Oy$:
\[
\ddot{y}^S(t) = -g.
\]
We integrate once and use the initial condition $v_{y,0}^S = v_0\sin\alpha$
\[
\dot{y}^S(t) = -gt + v_0\sin\alpha.
\]
And we integrate and use the initial condition $y_0^S = 0$ again to find
\[
y^S(t)= -\frac{1}{2}gt^2 + (v_0\sin\alpha) t.
\]
\item Equations for the cheese projected onto $Ox$ and $Oy$ :
Using the same method as above, we find
\begin{displaymath}
\left\{ \begin{array}{l}
\ddot{x}^C(t) = 0 \\
\ddot{y}^C(t) = -g 
\end{array}\right.\qquad
\left\{ \begin{array}{l}
\dot{x}^C(t) = 0 \\
\dot{y}^C(t) = -gt
\end{array}\right.\qquad
\left\{ \begin{array}{l}
{x}^C(t) = L \\
{y}^C(t) = -\frac{1}{2}gt^2 +H. \\
\end{array}\right.
\end{displaymath}\\
\end{itemize} 

\item The cheese and the stone do collide if there is a time $t=t_{coll}$ at which they have the same position, that is, if the following conditions are met:
\[ 
x^S(t_{coll}) = x^C(t_{coll})\qquad \textrm{ET} \qquad y^S(t_{coll}) = y^C(t_{coll}). 
\]\\
The condition $x^S(t_{coll}) = x^C(t_{coll})$ gives
\begin{equation}
(v_0\cos\alpha) t_{coll} = L \Rightarrow t_{coll} = \frac{L}{v_0\cos\alpha} = \frac{\sqrt{L^2+H^2}}{v_0}.
\label{eq_cond1}
\end{equation}
While the condition $y^S(t_{coll}) = y^C(t_{coll})$ gives
\begin{equation}
-\frac{1}{2}gt_{coll}^2 + (v_0\sin\alpha) t_{coll} = -\frac{1}{2}gt_{coll}^2 +H \Rightarrow t_{coll} = \frac{H}{v_0\sin\alpha} = \frac{\sqrt{L^2+H^2}}{v_0}.
\label{eq_cond2}
\end{equation}
Here, we used $\tan\alpha = \frac{\sin\alpha}{\cos\alpha} = \frac{H}{L}$.
The solutions (\ref{eq_cond1}) and (\ref{eq_cond2}) are identical: there is, indeed, a cheese-stone collision at time 
\begin{equation}\label{eq_tcoll} t_{coll}=\frac{\sqrt{L^2+H^2}}{v_0}. 
\end{equation}
We find that the expression for $t_{coll}$ does not depend on $g$: $g$ accelerates both objects vertically and in the same fashion (mathematically, that's the term in $-\frac{1}{2}g t^2$), which has no influence on the horizontal motion of the objects, and so none on $t_{coll}$ either. 
\item[b)] We just proved that the the stone and the cheese will always collide, no matter the initial speed $v_0$. Isn't that surprising? However, if $v_0$ is too small, the cheese will hit the ground before the collision happens: we must restrict $v_0$ in that respect. This is not contradicting the previous statement: the collision would just ``happen'' at $y<0$, that is, ``in the ground''. We would just have to dig a big enough hole to be able to see it.

In order for the collision to happen above ground, we must ensure $y^C(t_{coll})>0$, that is
\[
{y}^C(t_{coll}) =  -\frac{1}{2}gt_{coll}^2 +H > 0
\]
\[
\Rightarrow H>\frac{1}{2}gt_{coll}^2 = \frac{1}{2}g\frac{L^2+H^2}{v_0^2}
\]
\begin{equation}\label{eq_condv0}
\Rightarrow v_0 > \sqrt{g\frac{L^2+H^2}{2H}}.
\end{equation}

\item[c)]
Let's check that the results we found are in the right dimensions.
\begin{itemize}
\item Equation \ref{eq_tcoll}: $t_{coll}=\frac{\sqrt{L^2+H^2}}{v_0}:\frac{\left([m]^2\right)^{1/2}}{[m/s]}=\frac{[\cancel{m}]}{[\cancel{m}/s]}=[s]$. We find units of time.
\item Equation \ref{eq_condv0}: $v_0 > \sqrt{g\frac{L^2+H^2}{2H}}:\left([m/s^2]\frac{[m]^2}{[m]}\right)^{1/2}=\frac{\cancel{[m]^{1/2}}[m]}{[s]\cancel{[m]^{1/2}}}$. We find units of speed.
\end{itemize}
Let's see a few limiting cases to check if we find what we expect.
\begin{itemize}
\item If $v_0$ tends towards infinity, $t_{coll}$ must tend to 0 (Eq.\ref{eq_tcoll}) and the collision will happen above ground (Eq.\ref{eq_condv0})
\item If $v_0$ tends to 0, $t_{coll}$ must tend towards infinity (Eq.\ref{eq_tcoll}) and the collision cannot happen above ground since Eq.\ref{eq_condv0} is never satisfied.
\item If $g$ tends to 0, the right-hand side of Eq.\ref{eq_condv0} tends to 0, so the collision will happen above ground, whatever the speed $v_0>0$. 
\item If $H$ tends to 0, the right-hand side of Eq.\ref{eq_condv0} tends towards infinity, so the collision cannot happen above ground. 

\end{itemize}

\end{enumerate}



